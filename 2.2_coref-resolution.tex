Coreference resolution is the task of finding all expressions that refer to the same event and entity in a text. It is an important step for a lot of higher-level NLP tasks that involve natural language understanding such as document summarization, question answering, and information extraction. Recent research has introduced innovative approaches to sentence-level coreference resolution~\cite{grenander2023sentenceincremental,10.1162/tacl_a_00543} and document-level coreference resolution~\cite{10.1145/3539597.3573038,10.1007/978-3-031-40286-9_34}. The primary technology entails detecting candidate mentions, encoding them into vector representations, and identifying coreference relations by employing a Multilayer Perceptron (MLP) classifier to process the representations of each mention and past entities~\cite{Liu2023il}. Notably,~\citet{joshi2020spanbert} established the SpanBERT model, achieving state-of-the-art performance in coreference resolution. Furthermore,~\citet{yao-etal-2023-learning} developed a model inspired by SpanBERT, designed to learn and integrate multiple representations from both event alone and event pair. 

We implement coreference resolution at both the single-document and cross-document levels. For single-document coreference resolution, we employ F-coref~\cite{otmazgin2022fcoref} to identify the same entities on a sentence-by-sentence basis. Subsequently, for the cross-document coreference resolution, we utilize CDLM~\cite{caciularu-etal-2021-cdlm-cross} to extract coreferent entities. This crucial step serves to establish connections between identical entities within the knowledge graphs. Moreover, it enables tracking the specific document in which these entities are mentioned.

