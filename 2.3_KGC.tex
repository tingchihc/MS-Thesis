In knowledge graphs (KGs), latent relationships between events and entities often result in unknown connections, necessitating approaches to comprehend the missing information; this gap is addressed through knowledge graph completion~\cite{SHEN2022109597,Lin_Liu_Sun_Liu_Zhu_2015,Shi_Weninger_2018}, a method that predicts and fills in missing relationships or edges between entities, thereby enhancing the overall comprehensiveness of the knowledge graph.

~\citet{NIPS2013_1cecc7a7} invented TransE to construct entity and relation embeddings by treating relations as translations from head entity to tail entity. Inspired by~\citet{mikolov2013distributed}, TransE learns vector embeddings for entities and relationships, placing them in $\mathbb{R}^k$. The fundamental concept behind TransE is that the relationship between two entities is akin to a translation between their embeddings, expressed as $h+r\approx t$ when $(h,r,t)$ holds, where $h$ represents the head entity, $t$ represents the tail entity and $r$ represents the relationship from the head entity to tail entity, respectively. However, due to challenges in modeling 1-to-N, N-to-1, and N-to-N relations,~\citet{10.5555/2893873.2894046} introduced TransH to allow entities to have distinct representations in different relations. Incorporating BERT~\cite{devlin2019bert} into the knowledge graph completion task,~\citet{yao2019kgbert} developed KG-BERT. This model takes the entity and relation descriptions of a triple as input, computes the scoring function for the triple using the KG-BERT language model, and predicts unknown relationships.~\citet{zhang2023making}  incorporate the helpful KGs structural information into the LLMs, aiming to achieve structural-aware reasoning in the LLMs. They first transfer the existing LLMs paradigms to structural-aware settings and propose a knowledge prefix adapter to predict the unknown relationships.

To address this challenge in KGs, we utilize Vicuna~\cite{vicuna2023} to identify hidden relationships within each claim. Furthermore, we incorporate additional claims to elucidate latent connections in the KGs, enhancing the information within knowledge graphs. This process aids in a more comprehensive understanding of events and entities during the generation of multimodal multi-document claims.
